\begin{frame}
  \frametitle{Physical Model}
      \begin{itemize}
      \item{``Double'' diffusion $\Rightarrow$ both heat and chemical (solute) species are diffusing}
      \item{Heat and solute have competing affects on background fluid density}
      \item{Ex: Heated Salt Water
	\begin{itemize}
	\item{Warm water is slightly less dense (rises)}
	\item{Salty water is slightly more dense (falls)}
	\item{Natural convection ``cells'' can develop }
	\end{itemize}
      }
      \end{itemize}
\end{frame}

\begin{frame}[t]
  \frametitle{Governing Equations}
  \begin{eqnarray}
    \label{eqn:p_strong}
    \nabla \cdot \alert<2>{\bv{b}} - \nabla^2 p &=& 0 \\
    \label{eqn:T_strong}
    \D{T}{t} + \left( \alert<2>{\bv{b}}-\nabla p\right) \cdot \nabla T - \nabla^2{T} &=&0 \\
    \label{eqn:S_strong}
    \frac{\phi}{\sigma} \D{S}{t} +
    \left( \alert<2>{\bv{b}}-\nabla p\right) \cdot \nabla S -  \alert<3>{\kappa} \nabla^2{S}&=&0 
  \end{eqnarray}

  \only<2>
  {
    \begin{equation}
      \nonumber
      \bv{b} = ( \kappa R_S S - R_T T ) \hat{e}_{g}
    \end{equation}
  }

  \only<3>
  {
    \begin{eqnarray}
      \nonumber
      \kappa &=& \frac{\text{Solute Diffusivity}}{\text{Thermal Diffusivity}} \ll 1 \\
      \nonumber
      &=& Le^{-1}
    \end{eqnarray}
  }
\end{frame}


%% \begin{frame}[t]
%%   \frametitle{Stabilization}
%%   \begin{itemize}
%%   \item<1->
%%     {
%%       For small $\kappa$ on coarse grids, the Galerkin method is known to perform poorly.
%%     }
%%   \item<2->
%%     {
%%       We add an additional stabilizing term:
%%     }
%%   \end{itemize}
%%   \visible<1->
%%   {
%%     \begin{equation}
%%       \nonumber
%%       \int_{\Omega} \left(  \frac{\phi}{\sigma}\frac{\partial S^h}{\partial t} +
%%       \left( \bv{b}^h -\nabla p^h \right) \cdot \nabla S^h  \right) v^h
%%       + \kappa \nabla S^h \cdot \nabla v^h \;dx
%%       \end{equation}
%%       \visible<2->
%%       {
%% 	\begin{equation}
%% 	  \small
%%       \nonumber
%%       + \underbrace{\int_{\Omega'} \tau \left( \left(\bv{b}^h - \nabla p^h\right) \cdot \nabla v^h \right) \left[
%% 	\frac{\phi}{\sigma} \D{S^h}{t} +
%% 	\left( \bv{b}^h-\nabla p^h\right) \cdot \nabla S^h - \kappa \nabla^2 S^h \right]dx
%%       }_{\text{Stabilization Term}}
%%       \end{equation}
%%       }
%%       \begin{equation}
%%       \nonumber
%%       %\label{eqn:S_stabilized}
%%       =\int_{\partial \Omega} \kappa \left( \nabla S \cdot \hat{n} \right) v^h \; ds
%%     \end{equation}
%%   }
%% \end{frame}

\begin{frame}
  \frametitle{Solute and Temperature Contours}
  \vspace{-.25in}
  \begin{center}
    \begin{tabular}{cc} \\
      %\includegraphics[viewport=80 105 580 665,clip=true,angle=-90,width=0.45\textwidth]{figures/s_150x150_kappa_0_03}&
      %\includegraphics[viewport=80 105 580 665,clip=true,angle=-90,width=0.45\textwidth]{figures/t_150x150_kappa_0_03}\\
    \includegraphics[viewport=80 105 580 665,clip=true,angle=-90,width=0.45\textwidth]%
		    {figures/s_150x150_kappa_0_03_PRESENTATION}&
    \includegraphics[viewport=80 105 580 665,clip=true,angle=-90,width=0.45\textwidth]%
		    {figures/t_150x150_kappa_0_03_PRESENTATION}\\
    Solute Contours & Temperature Contours
    \end{tabular}
  \end{center}
\end{frame}

%% \begin{frame}
%%   \frametitle{Representative Adapted Grid}
%%   \vspace{-.25in}
%%   \begin{center}
%%       \includegraphics[viewport=80 105 580 665,clip=true,angle=-90,width=0.75\textwidth]{figures/grid_adaptive_kappa_0_03}
%%   \end{center}
%% \end{frame}

\begin{frame}[t]
  \frametitle{Sherwood Number}
  \begin{itemize}
  \item {The total solute (or heat) flux to a wall is an important engineering quantity of interest.}
  \item {The Sherwood number is defined as:
    \begin{equation}
      \nonumber
      N_S := -\int_{\partial \Omega_D} \!\!\!\!\!\!\! \nabla S \cdot n \;dx 
    \end{equation}
    }
  \item{For large $R_T, Le,$ it is known that 
    \begin{equation}
      \nonumber
      N_S \approx \sqrt{R_T Le} = \sqrt{\frac{R_T}{\kappa}}
    \end{equation}
  }
  \end{itemize}
\end{frame}




\begin{frame}
  \frametitle{Sherwood Number Computations}
  \vspace{-.25in}
  \begin{center}
      \includegraphics[viewport=10 10 560 725 ,clip=true,angle=-90,width=0.8\textwidth]{figures/Nu_vs_Le}
  \end{center}
\end{frame}

\begin{frame}
      \begin{itemize}
      \item{
	The problem of determining the Sherwood number can be thought of generically.}

      \item{
	We consider the integrated flux $I$ of some variable $u$ through
	a boundary $\Gamma \subseteq \partial \Omega$

	\begin{equation}
	  \nonumber
	  I := \int_{\Gamma} k \nabla u \cdot n \;ds = -\int_{\Gamma} \sigma \cdot n \;ds 
	\end{equation}
      }
        
      \item{For example, $I$ could be total mass/thermal flux, Lift, Drag, etc.}

      \item{Differentiating the FE solution
	%to obtain $\sigma\cdot n$ is only
	%$\mathcal{O}(h^p)$ accurate.
	generally yields 
	\begin{equation}
	  \nonumber
	  \| \sigma - \sigma_h \|_{L_2(\Gamma)} \approx \mathcal{O}(h^p)
	\end{equation}
      } 
	
      \end{itemize}
\end{frame}


\begin{frame}
  \frametitle{S.~C.~Boundary Flux Literature}
      \begin{itemize}
      \item
	{
          Many schemes for computing superconvergent boundary fluxes have been investigated:

	  \begin{itemize}
	  \item{J.~Douglas, T.~Dupont, and M.~Wheeler (1974)}
	  \item{M.~Wheeler (1979)}
	  \item{Carey et al. (1982, 2002)}
	  \item{Hughes, Larson, Cochburn, Wahlbin, Babu{\v{s}}ka, \ldots}
	  \end{itemize}
	}
	
      \item{Under some restrictions
	\begin{equation}
	  \nonumber
	  \| \sigma - \sigma_h \|_{L_2(\Gamma)} \approx \mathcal{O}(h^{p+1})
	\end{equation}

	is generally possible, see e.g.\ Douglas.}
	
      \end{itemize}
\end{frame}


\begin{frame}
  \frametitle{Model Problem: Poisson}
	\begin{equation}
	  \nonumber
	  -\nabla \cdot \left( k \nabla u \right)=f
	\end{equation}
      \begin{itemize}
      \item<1->{For all $\phi_i \neq 0$ on $\partial \Omega$
      \begin{equation}
	\nonumber
	-\int_{\partial \Omega} \left( \sigma\only<2->{\alert<2>{_h}} \cdot n\right)\phi_i \;ds =
	\int_{\Omega} k\nabla u\only<2->{\alert<2>{_h}} \cdot \nabla \phi_i - f\phi_i \;dx 
	\end{equation}

      }
	
      \item<3->{Can use this relation to solve for improved flux approximation.}

      \item<4->{Difficulties \ldots
	\begin{itemize}
	  \item {$\sigma \cdot n$ discontinuous at a node}
	  \item {$n$ not well-defined at nodes}
	  \item {3D extension}
	\end{itemize}
      }
      \end{itemize}
\end{frame}




\begin{frame}
      \begin{itemize}
      \item{
        It's simpler if we don't care about pointwise values of $\sigma \cdot n$,
	i.e.\ only the integrated flux is
	important.}

      \item{Carey (2002) uses partition of unity method: on $\partial \Omega$,
	\begin{equation}
	  \nonumber
	  \sum_{i=1}^N \phi_i = 1
	\end{equation}
	where $N$ is the number of nodes on $\partial \Omega$, to show that
	\begin{eqnarray}
	  \nonumber
	  -\int_{\partial \Omega} \left(\sigma_h \cdot n\right)
	   \underbrace{\left(\sum_{i=1}^N \phi_i \right)}_{=1} ds &=&
	\sum_{i=1}^N \int_{\Omega} k\nabla u_h \cdot \nabla \phi_i - f\phi_i \;dx
	%\\
	%I &=& \sum_{i=1}^N \int_{\Omega} k\nabla u_h \cdot \nabla \phi_i - f\phi_i \;dx 
	\end{eqnarray}
      }
      \end{itemize}
\end{frame}




\begin{frame}
      \begin{itemize}
      \item{
        We can do something similar on $\Gamma \subset \partial \Omega$: let $S$ be the set of elements
	which have a node on $\Gamma$, and let
	\begin{equation}
	  \nonumber
	  \sum_{i=1}^M \phi_i = 1 
	\end{equation}
	on $\Gamma$ where $M$ is the number of nodes on $\Gamma$.
	%, and $\Gamma_{+}$ is a slightly
	%larger domain.
      }
	
      \item{
	Then we have
	\begin{eqnarray}
	  \nonumber
	  \sum_{i=1}^M \int_{\Gamma_{+}}-\left(\sigma_h\cdot n\right)  \phi_i \;ds =
	  \sum_{i=1}^M \int_{S} k\nabla u_h \cdot \nabla \phi_i - f\phi_i \;dx 
	\end{eqnarray}
      }
      \end{itemize}
\end{frame}


\begin{frame}
  \begin{center}
    \includegraphics[width=.4\textwidth,angle=-90]{figures/square}    
  \end{center}

      \begin{itemize}
      \item{
        Example: $\Gamma$ is the bottom side, $\Gamma_{+}$ extends into the first element
	of the two adjacent sides.
	}
      \end{itemize}
\end{frame}





\begin{frame}
      \begin{itemize}
      \item{We can rewrite this as, let $ q_h  := -\left(\sigma_h \cdot n\right)$
	\begin{eqnarray}
	  \nonumber
	  \sum_{i=1}^M \int_{\Gamma} q_h \phi_i ds +
	  \int_{\Gamma_{+} \setminus \Gamma} \!\!\!\!\!\!\! q_h \phi_1 ds +
	  \int_{\Gamma_{+} \setminus \Gamma} \!\!\!\!\!\!\! q_h \phi_M ds 
	  &=&\\
	  \nonumber
	  \sum_{i=1}^M \int_{S} k\nabla u_h \cdot \nabla \phi_i - f\phi_i \;dx 
	\end{eqnarray}
	or,
	\begin{eqnarray}
	  \nonumber
	  I_h  &=&
	  \sum_{i=1}^M \int_{S} k\nabla u_h \cdot \nabla \phi_i - f\phi_i \;dx -
	  \int_{\Gamma_{+} \setminus \Gamma} \!\!\!\!\!\!\! q_h \phi_1 ds -
	  \int_{\Gamma_{+} \setminus \Gamma} \!\!\!\!\!\!\! q_h \phi_M ds 
	\end{eqnarray}
	}

      \item{If $\Gamma_{+} \setminus \Gamma$ is a Neumann boundary, the extra terms on the r.h.s.\
	are known.  In general the extra terms aren't known.}
      \end{itemize}
\end{frame}

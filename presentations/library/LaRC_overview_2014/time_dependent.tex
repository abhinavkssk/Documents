\subsection*{Time-Dependent Problems}
\begin{frame}%[t]
  \only<1>{
  }
%%   \begin{block}{}
%%     The weighted residual statement provides the connection between the mathematical
%%     statement of the problem and the computer code implementation of the problem:
%%   \end{block}

  %\begin{block}{}
  \begin{itemize}
    \only<1>
	{
	\item{For linear problems, we have already seen how
	  the weighted residual statement
	  leads directly to a sparse linear system of equations
	  \begin{equation}
	    \nonumber
	    \bv{K} \bv{U} = \bv{F}
	  \end{equation}
	  %which can be solved via Krylov subspace iterative methods.
	}
	}
    \only<2>
	{
	\item{For time-dependent problems, 
	  \begin{equation}
	    \nonumber
	    \frac{\partial u}{\partial t} = F(u)
	  \end{equation}
	}
	\item{we also need a way to advance the
	  solution in time, e.g. a $\theta$-method
	  \begin{eqnarray}
	    \nonumber
	    \left( \frac{ u^{n+1} - u^n}{\Delta t}, v^h\right) &=& \left(F(u_{\theta}), v^h\right)
	    \hspace{.1in} \forall v^h \in \mathcal{V}^h
	    %+ \mathcal{O}(\Delta t^{p(\theta)})
	    \\ \nonumber
	    u_{\theta} &:=& \theta u^{n+1} + (1-\theta)u^n
	  \end{eqnarray}
	\item{Leads to $\bv{K} \bv{U} = \bv{F}$ at \emph{each timestep}.}
	}
	}
  \end{itemize}
%\end{block}
\end{frame}





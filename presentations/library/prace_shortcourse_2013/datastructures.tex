\section{Key Data Structures}

%%%%%%%%%%%%%%%%%%%%%%%%%%%%%%%%%%%%%
\subsection{The Mesh Class}
\begin{frame}[shrink]
  \frametitle{The Mesh}
  \lstinputlisting{snippets/mesh.cxx}
\end{frame}

\begin{frame}[shrink]
  \frametitle{The Mesh}
  \lstinputlisting[language=bash]{snippets/mesh.cxx.out}
\end{frame}

\begin{frame}
  \frametitle{Operations on Objects in the \texttt{Mesh}}
  \begin{block}{}
    \begin{itemize}
    \item From a \texttt{Mesh} it is trivial to access ranges of objects of interest through \emph{iterators}.
    \item Iterators are simply a mechanism for accessing a range of objects.
    \item \libMesh{} makes extensive use of \emph{predictated iterators} to access, for example,
      \begin{itemize}
        \item All elements in the mesh.
        \item The ``active'' elements in the mesh assigned to the local processor in a parallel simulation.
        \item The nodes in the mesh.
      \end{itemize}
  \end{itemize}
  \end{block}
\end{frame}

\begin{frame}[shrink]
  \frametitle{Mesh Iterators}
  \lstinputlisting{snippets/active_elem_iterators.cxx}
\end{frame}

\begin{frame}[shrink]
  \frametitle{Mesh Iterators}
  \lstinputlisting{snippets/node_iterators.cxx}
\end{frame}



%%%%%%%%%%%%%%%%%%%%%%%%%%%%%%%%%%%%%
\subsection{The EquationSystems Class}
\begin{frame}
  \frametitle{EquationSystems}
  \begin{block}{}
    \begin{itemize}
      \item The \texttt{Mesh} is a discrete representation of the geometry for a problem.
      \item For a given \texttt{Mesh}, there can be an \texttt{EquationSystems} object, which represents one or more coupled system of equations posed on the \texttt{Mesh}.
        \begin{itemize}
          \item There is only one \texttt{EquationSystems} object per \texttt{Mesh} object.
          \item The \texttt{EquationSystems} object can hold many \texttt{System} objects, each representing a logical system of equations.
        \end{itemize}
      \item High-level operations such as solution input/output is usually handled at the \texttt{EquationSystems} level.
    \end{itemize}
  \end{block}
\end{frame}

\begin{frame}[shrink]
  \frametitle{EquationSystems}
  \lstinputlisting{snippets/es.cxx}
\end{frame}

\begin{frame}[shrink]
  \frametitle{EquationSystems}
  \lstinputlisting[language=bash]{snippets/es.cxx.out}
\end{frame}




%%%%%%%%%%%%%%%%%%%%%%%%%%%%%%%%%%%%%
\subsection{The Elem Class}
\begin{frame}
  \frametitle{Elements}
  \begin{block}{}
    \begin{itemize}
      \item The \texttt{Elem} base class defines a geometric element in \libMesh{}.
      \item An \texttt{Elem} is defined by \texttt{Node}s, Edges (2D,3D) and Faces (3D).
      \item An \texttt{Elem} is sufficiently rich that in many cases it is the only argument required to provide to a function.
    \end{itemize}
  \end{block}
\end{frame}

\begin{frame}[shrink]
  \frametitle{Elements}
  \lstinputlisting{snippets/elem.cxx}
\end{frame}
